\documentclass [a4paper, 11pt] {article}

\newcommand\sessiontitle{CDMA and spread spectrum techniques}
\newcommand\sessionnumber{2}
%\def\AvecSolutions{}  % Commenter pour ne pas afficher les solutions

\usepackage[left=25mm, right=25mm, top=20mm,
bottom=20mm]{geometry}
% \oddsidemargin =0 mm
% \topmargin = -10 mm
% \footskip = 20mm
% \textheight = 240 mm
% \textwidth = 160mm
\usepackage[utf8]{inputenc}
\usepackage[T1]{fontenc}
\usepackage{lmodern}
\usepackage[english]{babel}
\usepackage{fancybox}
\usepackage{listings}
\usepackage{color}
\usepackage{tikz}
\usetikzlibrary{babel}
\usetikzlibrary{decorations.markings}
\usepackage{pgfplots}
\pgfplotsset{compat=1.17}
\pgfplotsset{samples=200}
\usepackage{graphicx}
\usepackage{caption,subcaption}
\usepackage[titletoc]{appendix}
\usepackage{float} % figures flottantes
\usepackage{here} % figures flottantes
\usepackage{url}
\usepackage{enumitem}
\setlist[itemize]{label={$\bullet$}}
%\setlist[enumerate]{noitemsep, nolistsep}
\usepackage{xcolor}
\usepackage[colorlinks=true]{hyperref}
\usepackage{tabularx}
\usepackage{minted}
\usepackage{amsmath,amsfonts,amsthm,amsfonts,amssymb}
\usepackage[skins,breakable]{tcolorbox}
\usepackage{verbatim}
\usepackage[europeanresistors,siunitx]{circuitikz}
\usepackage{multicol}
\usepackage{physics}
\usepackage[outline]{contour} % glow around text
%\usetikzlibrary{intersections}
%\usetikzlibrary{decorations.markings}
\usetikzlibrary{angles,quotes} % for pic
\usetikzlibrary{bending} % for arrow head angle
\contourlength{1.0pt}
\usetikzlibrary{3d}
\usetikzlibrary{shapes,angles}
\usetikzlibrary{trees}
\usepackage{dirtytalk}

% --------------------------------------------------------------
% Title
% --------------------------------------------------------------
\makeatletter
\newcommand\maintitle[1]{
    \quitvmode
    \hb@xt@\linewidth{
        \dimen@=1ex
        \advance\dimen@-2pt
        \leaders\hrule \@height1ex \@depth-\dimen@\hfill
        \enskip
        \textbf{#1}
        \enskip
        \leaders\hrule \@height1ex \@depth-\dimen@\hfill
    }
}
\makeatother

\newcommand{\makesessiontitle}{
\begin{center}
    \Large
    \centering
    \maintitle{LELEC2796 - Wireless Communications}\\
    \textsc{\textbf{Session \sessionnumber{} - \sessiontitle{}}}\\
    \vspace{0.1cm}
    \normalsize
    Teachers: Claude Oestges and Luc Vandendorpe\\ Teaching Assistant: Jérome Eertmans\\
   \noindent\hrulefill
\end{center}
\thispagestyle{empty}
}

\usepackage{fancyhdr}
\pagestyle{fancy}
\fancyhf{}
\fancyhead[L]{Session \sessionnumber{}}
\fancyhead[C]{\sessiontitle}
\fancyhead[R]{LELEC2796}
\fancyfoot[C]{Page \thepage}

\fancypagestyle{firstpage}
{
   \cfoot{Page \thepage}
}
\fancypagestyle{nextpages}
{
    \lhead{Session \sessionnumber{}}
    \chead{\sessiontitle}
    \rhead{LELEC2796}
    \cfoot{Page \thepage}
}
% --------------------------------------------------------------
% exercise environments
% --------------------------------------------------------------
\newcounter{reminder}
\newcounter{exercise}
\newcounter{solution}

\counterwithin{equation}{exercise}
\counterwithin{equation}{solution}

\definecolor{exercise_color}{RGB}{21,76,121}
\definecolor{exercise_color_fill}{RGB}{252,248,227}

\newcommand{\theexerciseref}{No reference}
\newcounter{reminderequationcounter}
\newcounter{exerciseequationcounter}
\newcounter{solutionequationcounter}


\makeatletter
\newenvironment{reminder}
{
\global\chardef\dc@currentequation=\value{equation}%
\let\c@equation\c@solutionequationcounter
\renewcommand{\theequation}{R\arabic{equation}}%
% comment the following line if you don't use hyperref
\renewcommand{\theHequation}{R\arabic{equation}}%
\refstepcounter{reminder}
\par\vspace{5pt}\noindent
{\Huge\textbf{Reminder}}%
\par\vspace{10pt}\noindent\rmfamily}%
{\par\noindent\hrulefill\vrule width10pt height2pt depth2pt\par\setcounter{equation}{\dc@currentequation}}
\makeatother

\makeatletter
\newenvironment{exercise}[2][\texorpdfstring{\unskip}{}]
{
\global\chardef\dc@currentequation=\value{equation}%
\let\c@equation\c@exerciseequationcounter
\renewcommand{\theequation}{E\arabic{exercise}.\arabic{equation}}%
% comment the following line if you don't use hyperref
\renewcommand{\theHequation}{E\arabic{exercise}.\arabic{equation}}%
\refstepcounter{exercise}
\def\@currentlabel{{#2}}
\label{ref-exercise-\theexercise}
\addcontentsline{toc}{subsubsection}{{#2} #1}
\noindent
\flushleft
\begin{tikzpicture}
    \draw[very thick,exercise_color] (0,0) -- ++(0,+7.5pt)
    -- ++(\textwidth,0) node[midway,above] {\textbf{Exercise #2: #1}}
    -- ++(0,-7.5pt);
\end{tikzpicture}
\vspace{-.3cm}
\begin{tcolorbox}[
    blanker,
    width=\textwidth,
    breakable]
}
{   \end{tcolorbox}
\setcounter{equation}{\dc@currentequation}
\vspace{-.4cm}
\flushleft
\begin{tikzpicture}
    \draw[very thick,exercise_color] (0,0) -- ++(0,-7.5pt)
    -- ++(\textwidth,0)
    -- ++(0,+7.5pt);
\end{tikzpicture}
}
\makeatother

\makeatletter
\ifcsname AvecSolutions\endcsname
\newenvironment{solution}
{
\global\chardef\dc@currentequation=\value{equation}%
\let\c@equation\c@solutionequationcounter
\renewcommand{\theequation}{S\arabic{solution}.\arabic{equation}}%
% comment the following line if you don't use hyperref
\renewcommand{\theHequation}{S\arabic{solution}.\arabic{equation}}%
\refstepcounter{solution}
\noindent
\begin{tcolorbox}[
    colframe=exercise_color,
    colback=exercise_color_fill,
    coltitle=exercise_color_fill,
    title=\centering\textbf{{\hypersetup{allcolors=white} Solution to exercise  \ref{ref-exercise-\theexercise}:}},
    breakable,
    width=\textwidth]
}
{   \end{tcolorbox}
\setcounter{equation}{\dc@currentequation}
}
\else
\newenvironment{solution}{\comment}{\endcomment}
\fi
\makeatother

% --------------------------------------------------------------
% Code environments
% --------------------------------------------------------------
\usemintedstyle{borland}
\providecommand*{\listingautorefname}{Listing}

\newenvironment{python}
{\VerbatimEnvironment
\begin{minted}[
linenos,
% fontfamily=courier,
fontsize=\normalsize,
xleftmargin=21pt,
]{python}}
{\end{minted}}

\newcommand\py[1]{\mintinline{python}{#1}}

\newcommand\la[1]{\mintinline{latex}{#1}}

% Vertical line in matrices

\makeatletter
\renewcommand*\env@matrix[1][*\c@MaxMatrixCols c]{%
  \hskip -\arraycolsep
  \let\@ifnextchar\new@ifnextchar
  \array{#1}}
\makeatother

% Double underline
\def\doubleunderline#1{\underline{\underline{#1}}}


% Multiple autoref: https://tex.stackexchange.com/a/183682

\makeatletter

% define a macro \Autoref to allow multiple references to be passed to \autoref
\newcommand\Autoref[1]{\@first@ref#1,@}
\def\@throw@dot#1.#2@{#1}% discard everything after the dot
\def\@set@refname#1{%    % set \@refname to autoefname+s using \getrefbykeydefault
    \edef\@tmp{\getrefbykeydefault{#1}{anchor}{}}%
    \xdef\@tmp{\expandafter\@throw@dot\@tmp.@}%
    \ltx@IfUndefined{\@tmp autorefnameplural}%
         {\def\@refname{\@nameuse{\@tmp autorefname}s}}%
         {\def\@refname{\@nameuse{\@tmp autorefnameplural}}}%
}
\def\@first@ref#1,#2{%
  \ifx#2@\autoref{#1}\let\@nextref\@gobble% only one ref, revert to normal \autoref
  \else%
    \@set@refname{#1}%  set \@refname to autoref name
    \@refname~\ref{#1}% add autoefname and first reference
    \let\@nextref\@next@ref% push processing to \@next@ref
  \fi%
  \@nextref#2%
}
\def\@next@ref#1,#2{%
   \ifx#2@ and~\ref{#1}\let\@nextref\@gobble% at end: print and+\ref and stop
   \else, \ref{#1}% print  ,+\ref and continue
   \fi%
   \@nextref#2%
}

\makeatother


\begin{document}

    \makesessiontitle

    \begin{reminder}


    Spread spectrum techniques are characterized by the use of a transmission bandwidth wider than the signal bandwidth. This feature provides resistance against interference/jamming and multipath propagation. The increase of bandwidth comes from the use of \textit{spreading codes}. These codes are designed to have an autocorrelation as close as possible to a Dirac pulse and a cross-correlation as low as possible. Each user is hence attributed a unique code in order to separate users. In summary, spread spectrum techniques enable \textit{code division multiple access} (CDMA).

    Among the different existing techniques, we will here focus on the \textit{direct sequence spectrum spreading} (DS/SS). The uplink DS/CDMA is depicted in \Autoref{fig:rake_tx,fig:rake_rx}.

    \begin{figure}[H]
        \centering
        \begin{tikzpicture}
    \node (I1) {$I_1(n)$};
    \node[below of=I1, node distance=3cm] (Ik) {$I_k(n)$};
    
    \path (I1) -- (Ik) node[midway] (middle) {};
    
    \node[right of=I1, draw, minimum width=1.2cm, minimum height=1cm, node distance=1.8cm] (up1) {$\uparrow N_c$};
    \node[right of=up1, draw, minimum width=1.2cm, minimum height=1cm, node distance=2cm] (a1) {$a_1(m)$};
    \node[right of=a1, draw, minimum width=1.2cm, minimum height=1cm, node distance=2cm] (p1) {$p(t)$};
    \node[right of=p1, draw, minimum width=1.2cm, minimum height=1cm, node distance=2cm] (c1) {$c_1(t)$};
    \draw[->, thick] (I1) -- (up1);
    \draw[->, thick] (up1) -- (a1);
    \draw[->, thick] (a1) -- (p1);
    \draw[->, thick] (p1) -- (c1);
    \draw[dashed] ([shift={(-.1,-.3)}]up1.south west) rectangle ([shift={(+.1,+.3)}]p1.north east) node[right] {$p_1(t)$};
    \draw[dashed] ([shift={(-.2,-.6)}]up1.south west) rectangle ([shift={(+.2,+.6)}]c1.north east) node[right] {$h_1(t)$};
    
    \node[right of=Ik, draw, minimum width=1.2cm, minimum height=1cm, node distance=1.8cm] (upk) {$\uparrow N_c$};
    \node[right of=upk, draw, minimum width=1.2cm, minimum height=1cm, node distance=2cm] (ak) {$a_k(m)$};
    \node[right of=ak, draw, minimum width=1.2cm, minimum height=1cm, node distance=2cm] (pk) {$p(t)$};
    \node[right of=pk, draw, minimum width=1.2cm, minimum height=1cm, node distance=2cm] (ck) {$c_k(t)$};
    \draw[->, thick] (Ik) -- (upk);
    \draw[->, thick] (upk) -- (ak);
    \draw[->, thick] (ak) -- (pk);
    \draw[->, thick] (pk) -- (ck);
    \draw[dashed] ([shift={(-.1,-.3)}]upk.south west) rectangle ([shift={(+.1,+.3)}]pk.north east) node[right] {$p_k(t)$};
    \draw[dashed] ([shift={(-.2,-.6)}]upk.south west) rectangle ([shift={(+.2,+.6)}]ck.north east) node[right] {$h_k(t)$};
    
    \path ([shift={(+.2,+.6)}]c1.north east) -- ++(1.1,0) coordinate (NW);
    \path ([shift={(+.2,-.6)}]ck.south east) -- ++(1.6,0) coordinate (SW);
    \draw[->,thick] (c1) -- (NW |- c1);
    \draw[->,thick] (ck) -- (NW |- ck);
    \draw (NW) rectangle (SW);
    \path (SW) -- (NW |- SW) node[midway] (midsouth) {};
    \draw[<-,thick] (midsouth.center) -- ++(0,-.5) node[below] {$n(t)$};
    \draw[->,thick] (SW |- middle) -- ++(0.5,0) node[right] {$r(t)$};
    
    \path (I1 |- middle.center) -- (SW |- middle.center) node[midway] {$\vdots$};
    
\end{tikzpicture}
        \caption{DS/CDMA - Transmitters}
        \label{fig:rake_tx}
    \end{figure}


    In \autoref{fig:rake_tx}, it is assumed that $K$ users are communicating with a receiver. User $k$ has to transmit a sequence of symbols $I_k(n)$. This sequence is first upsampled by a factor $N_c$ and then convolved with the spreading code $a_k(m)$ associated to the user $k$. The resulting sequence is then passed into the pulse shaping filter $p(t)$ and transmitted over the channel $c_k(t)$.


    \autoref{fig:rake_rx} illustrates how a receiver can be configured to properly detect the symbols of a user $k$. It is assumed that the low pass channel $c_k(t)$ consists of $L$ multipath components and can hence be expressed as $c_k(t) = \sum_{l=0}^{L-1}\alpha_{kl}\delta(t-\tau_{kl})$. Every branch first performs a matched filtering operation, followed by sampling. The sequence is then dispreaded by means of a convolution with $a^*_k(-m)$. After down sampling, the symbols are multiplied by a complex gain $\alpha^*_{km}$ in order to compensate for the corresponding multipath component. Since the structures consist of $M \leq L$ branches, the receiver deals with $M$ paths out of $L$. As a result, the structure (called \textit{rake receiver} in the literature) has a diversity order of $M$.


    \begin{figure}[H]
        \centering
        \begin{tikzpicture}
    \node (rt) {$r(t)$};
    
    \path (rt) -- ++(.8,0) coordinate (mrt);
    
    \node[above right of=mrt, draw, minimum width=1.2cm, minimum height=1cm, node distance=2cm] (p1) {$p(-t)$};
    \node[right of=p1, node distance=2cm] (split1) {};
    \node[right of=split1, draw, minimum width=1.2cm, minimum height=1cm, node distance=2cm] (a1) {$a_k(m)$};
    \node[right of=a1, draw, minimum width=1.2cm, minimum height=1cm, node distance=1.8cm] (up1) {$\downarrow N_c$};
    \node[right of=up1, draw, circle, node distance=1.8cm] (mul1) {$\times$};
    \draw[->, thick] (rt) -- (mrt) |- (p1);
    \draw[-, thick] (p1) -- ++(1,0) -- ([yshift=.5cm]split1.center) node[above] {$nT + m T_c + \tau_{k,1}$};
    \draw[->, thick] (split1.center) -- (a1);
    \draw[->, thick] (a1) -- (up1);
    \draw[->, thick] (up1) -- (mul1);
    \draw[dashed] ([shift={(-.2,-.8)}]p1.south west) rectangle ([shift={(+.3,+1)}]mul1.north east);
    \draw[<-,thick] (mul1) -- ++(0,-.7) node[below] {$\alpha_{k,1}^*$};
    
    \node[below right of=mrt, draw, minimum width=1.2cm, minimum height=1cm, node distance=2cm] (pm) {$p(-t)$};
    \node[right of=pm, node distance=2cm] (splitm) {};
    \node[right of=splitm, draw, minimum width=1.2cm, minimum height=1cm, node distance=2cm] (am) {$a_k(m)$};
    \node[right of=am, draw, minimum width=1.2cm, minimum height=1cm, node distance=1.8cm] (upm) {$\downarrow N_c$};
    \node[right of=upm, draw, circle, node distance=1.8cm] (mulm) {$\times$};
    \draw[->, thick] (rt) -- (mrt) |- (pm);
    \draw[-, thick] (pm) -- ++(1,0) -- ([yshift=.5cm]splitm.center) node[above] {$nT + m T_c + \tau_{k,m}$};
    \draw[->, thick] (splitm.center) -- (am);
    \draw[->, thick] (am) -- (upm);
    \draw[->, thick] (upm) -- (mulm);
    \draw[dashed] ([shift={(-.2,-.8)}]pm.south west) rectangle ([shift={(+.3,+1)}]mulm.north east);
    \draw[<-,thick] (mulm) -- ++(0,-.7) node[below] {$\alpha_{k,m}^*$};
    
    \path (mul1) -- ++(2,0) coordinate (X);
    
    \node[circle,draw] (add) at (X |- rt) {$+$};
    \draw[->,thick] (add) -- ++(1,0);
    
    \draw[->,thick] (mul1) -- ++(1,0) -- (add);
    \draw[->,thick] (mulm) -- ++(1,0) -- (add);

    
\end{tikzpicture}
        \caption{Structure of the rake receiver - order M.}
        \label{fig:rake_rx}
    \end{figure}

    \vspace{0.2 cm}

    \textbf{Reference:} Slides lectures 4 to 6.

    \end{reminder}

    \part*{Exercices}

    \begin{exercise}[Rake receiver]{1}

Suppose that the power delay profile of a multipath channel is given by
\begin{equation}
    P(\tau) = \frac{P_0}{\mu_\tau}e^{-\tau/\mu_\tau},
\end{equation}
and that DS/CDMA is employed as spread spectrum technique on this channel.

We will make the following assumptions:

\begin{itemize}
    \item the receiver employs a two-fingers rake receiver;
    \item each of the two channel taps is characterized by Rayleigh fading;
    \item the delays of arrival of the two taps are given by $T_c$ and $2T_c$, where $T_c$ is the chip duration;
    \item the transmitted symbols come from a BPSK constellation and have an energy given by $E_s$;
    \item the noise $w(t)$ is assumed to be AWGN, zero mean, stationary, and with bilateral constant power spectral density $N_0$;
    \item this noise is added to the signal just before the rake receiver (as in the reminder);
    \item the spreading sequences have normalized and ideal autocorrelation and cross correlation functions. The self-interference (including intersymbol) and the multiple access interference are neglected;
    \item and the pulse shaping filters are real, ideal and normalized. Please remember that an ideal filter satisfies the \href{https://en.wikipedia.org/wiki/Nyquist_ISI_criterion}{Nyquist ISI criterion}.
\end{itemize}


{\normalfont \textbf{Questions}}

\begin{enumerate}
    \item Derive the expressions of the average and RMS delay spreads of the channel.
    \item Give the expression of the final signal at the output of the rake receiver.
    \item In this final signal, what is the distribution of the term coming from the additive noise? Derive its mean and its variance, assuming $\alpha_1$ and $\alpha_2$ are constant.
    \item Derive the probability density function of the sum of the squared amplitudes of the two channel taps exploited by the rake receiver.
    \item Derive the expression of the bit error rate of the system.
\end{enumerate}

So far, we have assumed that the rake was performing a \textit{maximum ratio combining} operation. The signals at the output of each branch of the rake were all added. Let us now assume that the receiver performs \textit{selective combining}: instead of adding the signals of the two branches, the receiver now selects the best of the two branches.

\begin{enumerate}
    \setcounter{enumi}{5}
    \item Derive the probability density function of the maximum of the squared amplitudes of the two channel taps.
    \item Derive the expression of the bit error rate in this case.
    \item Plot the BER as function of the signal-to-noise ratio $E_s/N_0$ for both maximum ratio and selective combining. Compare your results.
\end{enumerate}

\newpage



\textbf{Hints}\\

The following hints and identities will be useful during the session:

\begin{itemize}

    \item Integration by parts:
    \begin{equation}\int f(x)\frac{\partial g(x)}{\partial x} dx = \big[f(x)g(x)\big] - \int \frac{\partial f(x)}{\partial x}g(x)dx. \end{equation}
    \item Theorem: let $N(t)$ be a real Gaussian process (with $\mathbb{E}[N(t)]$ and $\mathbb{E}[N(t)N(s)]$ being continuous functions in variables $s$ and $t$), and $f(t)$ be a real deterministic and continuous function. The random variable defined as
	\begin{equation}
	Z = \int_a^b N(t)f(t)dt,
	\end{equation}
	is a Gaussian random variable.
    \item[-] For BPSK symbols transmitted over an AWGN channel, the bit error rate can be expressed as

    \begin{equation}
        P_e(\rho) = \frac{1}{2}\text{erfc}\big(\sqrt{\rho}\big),
    \end{equation}
    where $\rho$ the ratio of the energy per symbol divided by the noise power density.

    \item[-] Let $h$ be a Rayleigh fading channel. In that case, we have
    \begin{itemize}
        \item $h \sim \mathcal{C}\mathcal{N}(0,\sigma)$,
        \item $|h| \sim \text{Rayleigh}(\sigma)$,
        \item and $|h|^2 \sim \exp(2\sigma^2)$ (in the beta parametrization).
    \end{itemize}

    \item[-] Let $X \sim \text{exp}(\beta)$ be an exponential random variable of parameter $\beta$.

    \begin{itemize}
    \item[-] the mean of X is equal to $\beta$
    \item[-] the probability density function of $X$ is given by $f_X(x) = \frac{1}{\beta}e^{-x/\beta}u(x)$
    \item[] (where $u(x)$ is the step function)
    \item[-] its cumulative distribution function is given by $F_X(x) = \big(1 - e^{-x/\beta}\big)u(x)$
    \item[-] its characteristic function is given by $\phi_X(t) = (1-\beta jt)^{-1}$.
    \end{itemize}
    \item Reminder: the characteristic function of the sum of random variables is the product of the characteristic functions of these random variables.
    \item[-] One identity involving the complementary error function:
    \begin{align}
    &\int^{+\infty}_{0}e^{-cx}\text{erfc}(\sqrt{ax}) dx = \frac{1}{c} \bigg[1-\sqrt{\dfrac{a}{a+c}}\bigg] \; \; \text{for} \; \; a,c \in \mathbb{R}^+.
    \end{align}

\end{itemize}
    \end{exercise}

    \begin{solution}
        \begin{enumerate}
    \item By using integration by parts, the mean and RMS delay spreads are given by

\begin{align}
        \tau_M &= \frac{\int_0^\infty \tau P(\tau) d\tau}{\int_0^\infty P(\tau) d\tau} = \hdots = \mu_{\tau}, \\
        \tau_{RMS} & = \sqrt{\frac{\int_0^\infty (\tau - \tau_M)^2 P(\tau) d\tau}{\int_0^\infty P(\tau) d\tau}} = \hdots = \mu_{\tau}.
\end{align}

\item The general expression of the signal obtained at the output of the rake receiver is provided in the slides of the lectures and is given by

\begin{align}
    y_{k'}(n') =& I_{k'}(n') \sum_{l'=0}^{M-1}|\alpha_{k'l'}|^2 +  \underbrace{I_{k'}(n')\sum_{l'=0}^{M-1}\alpha^*_{k'l'}\sum_{l\neq l'}^{L-1}\alpha_{k'l}C_{k'k'}\big[\tau_{k'l'}-\tau_{k'l} \big]}_{\text{Self interference}}\\
    &+ \underbrace{\sum_{l'=0}^{M-1} \alpha^*_{k'l'} \int_{-\infty}^{+\infty}w(t)p^*_{k'}(t-n'T-\tau_{k'l'})dt}_{\text{coming from noise}}\nonumber\\
    &+ \text{Intersymbol interference and Multiple access interference}. \nonumber
\end{align}

The self-interference (including intersymbol) and the multiple access interference are neglected. In addition, we are here considering a two-fingers rake. Based on these assumptions, the received signal can be simplified:

\begin{align}
    y_k(n) &= \underbrace{\Big(|\alpha_1|^2 + |\alpha_2|^2 \Big)}_{\text{Attenuation factor}} I_k(n)\\
    &+ \underbrace{\alpha^*_1 \int_{-\infty}^{+\infty} w(t)p_k^*(t-nT-\tau_{1})dt + \alpha^*_2 \int_{-\infty}^{+\infty} w(t)p_k^*(t-nT-\tau_{2})dt}_{\text{Equivalent noise}}, \nonumber
\end{align}

where $\alpha_1$ and $\alpha_2$ are the two taps of the channel and $p_k(t)$ is the concatenation of the pulse shaping filter with the spreading code.

\item By applying the theorem provided in the hints, we can observe that the two integrals are both Gaussian random variables. For fixed $\alpha_1$ and $\alpha_2$, the whole term is also Gaussian since a linear combination of Gaussian r.v. is a Gaussian r.v.

For fixed $\alpha_1$ and $\alpha_2$, the mean $m$ of this term is given by

\begin{align}
    m &= \mathbb{E}\Bigg[\alpha^*_1 \int_{-\infty}^{+\infty} w(t)p_k^*(t-nT-\tau_{1})dt + \alpha^*_2 \int_{-\infty}^{+\infty} w(t)p_k^*(t-nT-\tau_{2})dt \Bigg],\\ \intertext{(by linearity)}
    &= \alpha^*_1 \int_{-\infty}^{+\infty} \mathbb{E}\big[w(t)\big]p_k^*(t-nT-\tau_{1})dt \\
    &+ \alpha^*_2 \int_{-\infty}^{+\infty} \mathbb{E}\big[w(t)\big]p_k^*(t-nT-\tau_{2})dt = 0. \nonumber
\end{align}

The variance $\sigma^2$ of the term is given by

\begin{align}
\sigma^2 &= \mathbb{E}\Bigg[\bigg(\alpha^*_1 \int_{-\infty}^{+\infty} w(t)p_k^*(t-nT-\tau_{1})dt + \alpha^*_2 \int_{-\infty}^{+\infty} w(t)p_k^*(t-nT-\tau_{2})dt\bigg) \\
& \times \bigg( \alpha^*_1 \int_{-\infty}^{+\infty} w(t')p_k^*(t'-nT-\tau_{1})dt' + \alpha^*_2 \int_{-\infty}^{+\infty} w(t')p_k^*(t'-nT-\tau_{2})dt'\bigg)^*\bigg].\nonumber
\end{align}

By distributing the terms and by interchanging the expectation operator with the integrals, we obtain

\begin{align}
\sigma^2 &= |\alpha_1|^2 \int_{-\infty}^{+\infty}\int_{-\infty}^{+\infty}\mathbb{E}\big[w(t)w^*(t')\big]p_k^*(t-nT-\tau_{1})p_k(t'-nT-\tau_{1})dtdt' \\
&+ |\alpha_2|^2 \int_{-\infty}^{+\infty}\int_{-\infty}^{+\infty}\mathbb{E}\big[w(t)w^*(t')\big]p_k^*(t-nT-\tau_{2})p_k(t'-nT-\tau_{2})dtdt' \nonumber\\
&+ \alpha_1^*\alpha_2 \int_{-\infty}^{+\infty}\int_{-\infty}^{+\infty}\mathbb{E}\big[w(t)w^*(t')\big]p_k^*(t-nT-\tau_{1})p_k(t'-nT-\tau_{2})dtdt' \nonumber\\
&+ \alpha_2^*\alpha_1 \int_{-\infty}^{+\infty}\int_{-\infty}^{+\infty}\mathbb{E}\big[w(t)w^*(t')\big]p_k^*(t-nT-\tau_{2})p_k(t'-nT-\tau_{1})dtdt'\nonumber
\end{align}

By definition, the expectation $\mathbb{E}\big[w(t)w^*(t')\big]$ is the autocorrelation function of the noise, given by $N_0\delta(t-t')$. The expression of the variance becomes

\begin{align}
\sigma^2 &= |\alpha_1|^2N_0 \int_{-\infty}^{+\infty}\int_{-\infty}^{+\infty}\delta(t-t')p_k^*(t-nT-\tau_{1})p_k(t'-nT-\tau_{1})dtdt' \\
&+ |\alpha_2|^2N_0 \int_{-\infty}^{+\infty}\int_{-\infty}^{+\infty}\delta(t-t')p_k^*(t-nT-\tau_{2})p_k(t'-nT-\tau_{2})dtdt' \nonumber\\
&+ \alpha_1^*\alpha_2N_0 \int_{-\infty}^{+\infty}\int_{-\infty}^{+\infty}\delta(t-t')p_k^*(t-nT-\tau_{1})p_k(t'-nT-\tau_{2})dtdt' \nonumber\\
&+ \alpha_2^*\alpha_1N_0 \int_{-\infty}^{+\infty}\int_{-\infty}^{+\infty}\delta(t-t')p_k^*(t-nT-\tau_{2})p_k(t'-nT-\tau_{1})dtdt'\nonumber
\end{align}


The following properties of the Delta function can be used to simplify the integrals:

\begin{align}
    \int \delta(x-a) dx = 1, &&  f(x)\delta(x-a) = f(a)\delta(x-a).
\end{align}

We obtain

\begin{align}
\sigma^2 &= |\alpha_1|^2N_0 \int_{-\infty}^{+\infty}p_k^2(t'-nT-\tau_{1})dt' + |\alpha_2|^2N_0 \int_{-\infty}^{+\infty}p_k^2(t'-nT-\tau_{2})dt' \\
&+ \alpha_1^*\alpha_2N_0 \int_{-\infty}^{+\infty}p_k^*(t'-nT-\tau_{1})p_k(t'-nT-\tau_{2})dt'\nonumber\\
&+ \alpha_2^*\alpha_1N_0 \int_{-\infty}^{+\infty}p_k^*(t'-nT-\tau_{2})p_k(t'-nT-\tau_{1})dt'\nonumber
\end{align}

In the above expression, one make the change of variable $t''=t'-nT$ to simply the expression:

\begin{align}
\sigma^2 &= |\alpha_1|^2N_0 \int_{-\infty}^{+\infty}p_k^2(t''-\tau_{1})dt'' + |\alpha_2|^2N_0 \int_{-\infty}^{+\infty}p_k^2(t''-\tau_{2})dt'' \\
&+ \alpha_1^*\alpha_2N_0 \int_{-\infty}^{+\infty}p_k^*(t''-\tau_{1})p_k(t''-\tau_{2})dt''\nonumber\\
&+ \alpha_2^*\alpha_1N_0 \int_{-\infty}^{+\infty}p_k^*(t''-\tau_{2})p_k(t''-\tau_{1})dt'' \nonumber
\end{align}

The filter $p_k(t)$ is by definition the concatenation of the pulse shaping filter $p(t)$ with the spreading sequence $a_k(n)$:

\begin{equation}
    p_k(t) = \sum_{n=-\infty}^{+\infty}a_k(n)p(t-nT_c).
\end{equation}

Let us also recall that $\tau_1 = T_c$ and $\tau_2 = 2T_c$.
By substituting the $p_k(t)$ in the first term of the variance ($\triangleq\sigma_{11}^2$), we obtain

\begin{align}
    \sigma_{11}^2 &=|\alpha_1|^2N_0 \int_{-\infty}^{+\infty}p_k^2(t''-\tau_{1})dt'', \\
    &= |\alpha_1|^2N_0 \int_{-\infty}^{+\infty}\Big[ \sum_{n=-\infty}^{+\infty}a_k(n)p(t''- (n+1)T_c) \Big]^2 dt'',\\
    &= |\alpha_1|^2N_0 \int_{-\infty}^{+\infty}\Big[ \sum_{n=-\infty}^{+\infty}a^2_k(n)p^2(t''- (n+1)T_c) \Big] dt'' \\
    &+ 2|\alpha_1|^2N_0 \int_{-\infty}^{+\infty}\Big[\sum_{n=-\infty}^{+\infty}\sum_{n'\neq n}a_k(n)a_k(n')p(t''- (n+1)T_c) p(t''- (n'+1)T_c) dt''\Big] \nonumber,\\
    &= |\alpha_1|^2N_0 \underbrace{\Big[ \sum_{n=-\infty}^{+\infty}a^2_k(n) \Big]}_{1} \underbrace{\int_{-\infty}^{+\infty} p^2(t''- (n+1)T_c) \Big] dt''}_{1} \\
    &+ 2|\alpha_1|^2N_0 \underbrace{\Big[\sum_{n=-\infty}^{+\infty}\sum_{n'\neq n}a_k(n)a_k(n') \Big] \underbrace{\int_{-\infty}^{+\infty} p(t''- (n+1)T_c) p(t''- (n'+1)T_c) dt''}_{\delta(n-n')}}_{0}\nonumber,\\
    &= |\alpha_1|^2N_0.
\end{align}

In the last equality, we have used the following properties:

\begin{itemize}
    \item the autocorrelation of the training sequence is ideal and normalized: $\big[ \sum_{n=-\infty}^{+\infty}a^2_k(n)\big] = 1$,
    \item the pulse shaping filter is normalized, so its energy $\int p^2(t) dt$ is equal to 1,
    \item and the pulse shaping filter is such that $\int_{-\infty}^{+\infty} p(t''- (n+1)T_c) p(t''- (n'+1)T_c) dt'' = \delta(n-n')$.
\end{itemize}

The second term is equal to $|\alpha_2|^2N_0$ thanks to the same reasoning.

By substituting $p_k(t)$ in the third term, we obtain

\begin{align}
&\alpha_1^*\alpha_2N_0 \int_{-\infty}^{+\infty}\sum_{n=-\infty}^{+\infty}\sum_{n'=-\infty}^{+\infty}a_k(n)a_k(n')p(t''-\tau_1-nT_c)p(t''-\tau_2-n'T_c)dt'',\\
&= \alpha_1^*\alpha_2N_0 \sum_{n=-\infty}^{+\infty}\sum_{n'=-\infty}^{+\infty}a_k(n)a_k(n')\\
&\hspace{1cm} \times\underbrace{\int_{-\infty}^{+\infty}p(t''-(n+1)T_c)p(t''-(n'+2)T_c)dt''}_{\delta(n-n'-1)}\nonumber,\\
&= \alpha_1^*\alpha_2N_0 \underbrace{\sum_{n=-\infty}^{+\infty}a_k(n)a_k(n-1)}_{0},\\
&= 0.
\end{align}

The fourth term is also equal to zero for the same reasons.

We finally have
\begin{equation}
\sigma^2 = \Big(|\alpha_1|^2+|\alpha_2|^2\Big)N_0.
\end{equation}

Based on the above results, the received signal can be rewritten in a more compact manner

\begin{equation}
    y_k(n) = \Big(|\alpha_1|^2 + |\alpha_2|^2 \Big) I(n) + \tilde{w}(n),
\end{equation}

with $\tilde{w}(n) \sim \mathcal{N}\Big(0,\big(|\alpha_1|^2+|\alpha_2|^2\big)N_0\Big)$.

The instantaneous signal-to-noise ratio can hence be expressed as

\begin{equation}
    \text{SNR} = \dfrac{\Big(|\alpha_1|^2+|\alpha_2|^2\Big)^2E_s}{\Big(|\alpha_1|^2+|\alpha_2|^2\Big)N_0} = \Big(|\alpha_1|^2+|\alpha_2|^2\Big) \dfrac{E_s}{N_0}.
\end{equation}

Note that so far, the values of $\alpha_1$ and $\alpha_2$ have been assumed to be constant. As a result, the mean and the variance that have been computed are actually conditioned on $\alpha_1$ and $\alpha_2$. In the next question, we will take into account the fact that $\alpha_1$ and $\alpha_2$ are random variables.

\item Let $Z = \Big(|\alpha_1|^2+|\alpha_2|^2\Big)$ be the sum of the two squared amplitudes. Let us also define $X = |\alpha_1|^2$ and $Y = |\alpha_2|^2$. Since the fading of each tap is Rayleigh, we have

\begin{align}
    X &\sim \exp(\beta_1), \; \; \text{where} \; \; \beta_1 = \mathbb{E}\big[X\big] = \mathbb{E}\big[|\alpha_1|^2\big] = \frac{P_0}{\mu_\tau}e^{-T_c/\mu_\tau},\\
    Y &\sim \exp(\beta_2), \; \; \text{where} \; \; \beta_2 = \mathbb{E}\big[Y\big] = \mathbb{E}\big[|\alpha_2|^2\big] = \frac{P_0}{\mu_\tau}e^{-2T_c/\mu_\tau}.
\end{align}

Since $Z = X + Y$, the characteristic function of $Z$ is the product of the characteristic functions of $X$ and $Y$ (see hints):

\begin{align}
\phi_Z(t) &= \phi_X(t)\phi_Y(t),\\
&= \dfrac{1}{(1-\beta_1 jt)(1-\beta_2 jt)},\\ \intertext{(using partial fraction decomposition)}
&= \dfrac{\beta_1/(\beta_1-\beta_2)}{(1-\beta_1 jt)} + \dfrac{\beta_2/(\beta_2-\beta_1)}{(1-\beta_2 jt)}.
\end{align}

By inverting the characteristic function of $Z$, we can obtain its probability density function, given by

\begin{equation}
    f_Z(z)=\left\{
                \begin{array}{ll}
                  \dfrac{1}{\beta_1-\beta_2}\exp\Big( \dfrac{-z}{\beta_1}\Big) + \dfrac{1}{\beta_2-\beta_1}\exp\Big( \dfrac{-z}{\beta_2}\Big),\; \; \text{for} \; \; z > 0, \\
                  \\
                0, \; \; \text{for} \; \; z < 0.
                \end{array}
              \right.
\end{equation}

\item For one realization of the channel, the instantaneous BER is given by

\begin{equation}
    \frac{1}{2}\text{erfc}\Bigg(\sqrt{(|\alpha_1|^2+|\alpha_2|^2) \frac{E_s}{N_0}}\Bigg)
\end{equation}

This instantaneous BER must be integrated over the probability density function of $Z = |\alpha_1|^2+|\alpha_2|^2$. Using the identity provided in the hints, we obtain

\begin{align}
    P_e &= \int_{z}\frac{1}{2}\text{erfc}\Bigg(\sqrt{\frac{E_s}{N_0}z}\Bigg)f_Z(z) dz,\\
    &= \dfrac{1}{2}\dfrac{\beta_1}{\beta_1-\beta_2}\Bigg[ 1 - \sqrt{\dfrac{E_s/N_0}{E_s/N_0 + 1/\beta_1}} \Bigg] + \dfrac{1}{2}\dfrac{\beta_2}{\beta_2-\beta_1}\Bigg[ 1 - \sqrt{\dfrac{E_s/N_0}{E_s/N_0 + 1/\beta_2}} \Bigg].
\end{align}

\item Let $Z = \text{max}(X,Y)$, with $X = |\alpha_1|^2$ et $Y = |\alpha_2|^2$.

The cumulative distribution function of $Z$ is given by

\begin{align}
    F_Z(z) &= \mathbb{P}\big[ Z \leq z\big], \\
    &= \mathbb{P}\big[ X \leq z \; \text{and} \; Y \leq z\big], \\
    &= \mathbb{P}\big[ X \leq z\big]\mathbb{P}\big[ Y \leq z\big], \\
    &= \Big(1-e^{-z/\beta_1} \Big)\Big(1-e^{-z/\beta_2} \Big), \; \; \text{for} \; \; z > 0 \; \; \text{and} \; \; 0 \; \; \text{for} \; \; z < 0.
\end{align}

The probability density function of $z$ is hence given by

\begin{equation}
    f_Z(z) = \dfrac{\partial F_Z(z)}{\partial z}=\left\{
                \begin{array}{ll}
                 \dfrac{e^{-z/\beta_1}}{\beta_1}\Big(1-e^{-z/\beta_2} \Big) + \dfrac{e^{-z/\beta_2}}{\beta_2}\Big(1-e^{-z/\beta_1} \Big), \; \; \text{for} \; \; z > 0, \\
                  \\
                0, \; \; \text{for} \; \; z < 0.
                \end{array}
              \right.
\end{equation}

By distributing the factors, this expression can be rewritten as

\begin{equation}
    f_Z(z) = \dfrac{\partial F_Z(z)}{\partial z}=\left\{
                \begin{array}{ll}
                 \dfrac{e^{-z/\beta_1}}{\beta_1} + \dfrac{e^{-z/\beta_2}}{\beta_2} - \dfrac{e^{-z/\frac{\beta_1 \beta_2}{\beta_1+\beta_2}}}{\frac{\beta_1 \beta_2}{\beta_1+\beta_2}}, \; \; \text{for} \; \; z > 0, \\
                  \\
                0, \; \; \text{for} \; \; z < 0.
                \end{array}
              \right.
\end{equation}

\item For one realization of the channel, the instantaneous BER is given by

\begin{equation}\frac{1}{2}\text{erfc}\Bigg(\sqrt{\frac{E_s}{N_0}z} \Bigg),
\end{equation}

where $z = \text{max}\big(|\alpha_1|^2,|\alpha_2|^2\big)$.This instantaneous BER must be integrated over the probability density function of z. Using the identity provided in the hints, we obtain

\begin{align}
    P_e &= \int_{z}\frac{1}{2}\text{erfc}\Bigg(\sqrt{\frac{E_s}{N_0}z}\Bigg)f_Z(z) dz,\\
    &= \frac{1}{2} \Bigg[1-\sqrt{\dfrac{E_s/N_0}{E_s/N_0+ 1/\beta_1}}\Bigg]\\
    &+ \frac{1}{2} \Bigg[1-\sqrt{\dfrac{E_s/N_0}{E_s/N_0+ 1/\beta_2}}\Bigg] \nonumber\\
    &- \frac{1}{2} \Bigg[1-\sqrt{\dfrac{E_s/N_0}{E_s/N_0+\frac{\beta_1 + \beta_2}{\beta_1\beta_2}}}\Bigg].\nonumber
\end{align}



\end{enumerate}

    \end{solution}




\end{document}
