\usepackage[left=25mm, right=25mm, top=20mm,
bottom=20mm]{geometry}
% \oddsidemargin =0 mm
% \topmargin = -10 mm
% \footskip = 20mm
% \textheight = 240 mm
% \textwidth = 160mm
\usepackage[utf8]{inputenc}
\usepackage[T1]{fontenc}
\usepackage{lmodern}
\usepackage[english]{babel}
\usepackage{fancybox}
\usepackage{listings}
\usepackage{color}
\usepackage{tikz}
\usetikzlibrary{babel}
\usetikzlibrary{decorations.markings}
\usepackage{pgfplots}
\pgfplotsset{compat=1.17}
\pgfplotsset{samples=200}
\usepackage{graphicx}
\usepackage{caption,subcaption}
\usepackage[titletoc]{appendix}
\usepackage{float} % figures flottantes
\usepackage{here} % figures flottantes
\usepackage{url}
\usepackage{enumitem}
\setlist[itemize]{label={$\bullet$}}
%\setlist[enumerate]{noitemsep, nolistsep}
\usepackage{xcolor}
\usepackage[colorlinks=true]{hyperref}
\usepackage{tabularx}
\usepackage{minted}
\usepackage{amsmath,amsfonts,amsthm,amsfonts,amssymb}
\usepackage[skins,breakable]{tcolorbox}
\usepackage{verbatim}
\usepackage[europeanresistors,siunitx]{circuitikz}
\usepackage{multicol}
\usepackage{physics}
\usepackage[outline]{contour} % glow around text
%\usetikzlibrary{intersections}
%\usetikzlibrary{decorations.markings}
\usetikzlibrary{angles,quotes} % for pic
\usetikzlibrary{bending} % for arrow head angle
\contourlength{1.0pt}
\usetikzlibrary{3d}
\usetikzlibrary{shapes,angles}
\usetikzlibrary{trees}
\usepackage{dirtytalk}

% --------------------------------------------------------------
% Title
% --------------------------------------------------------------
\makeatletter
\newcommand\maintitle[1]{
    \quitvmode
    \hb@xt@\linewidth{
        \dimen@=1ex
        \advance\dimen@-2pt
        \leaders\hrule \@height1ex \@depth-\dimen@\hfill
        \enskip
        \textbf{#1}
        \enskip
        \leaders\hrule \@height1ex \@depth-\dimen@\hfill
    }
}
\makeatother

\newcommand{\makesessiontitle}{
\begin{center}
    \Large
    \centering
    \maintitle{LELEC2796 - Wireless Communications}\\
    \textsc{\textbf{Session \sessionnumber{} - \sessiontitle{}}}\\
    \vspace{0.1cm}
    \normalsize
    Teachers: Claude Oestges and Luc Vandendorpe\\ Teaching Assistant: Jérome Eertmans\\
   \noindent\hrulefill
\end{center}
\thispagestyle{empty}
}

\usepackage{fancyhdr}
\pagestyle{fancy}
\fancyhf{}
\fancyhead[L]{Session \sessionnumber{}}
\fancyhead[C]{\sessiontitle}
\fancyhead[R]{LELEC2796}
\fancyfoot[C]{Page \thepage}

\fancypagestyle{firstpage}
{
   \cfoot{Page \thepage}
}
\fancypagestyle{nextpages}
{
    \lhead{Session \sessionnumber{}}
    \chead{\sessiontitle}
    \rhead{LELEC2796}
    \cfoot{Page \thepage}
}
% --------------------------------------------------------------
% exercise environments
% --------------------------------------------------------------
\newcounter{reminder}
\newcounter{exercise}
\newcounter{solution}

\counterwithin{equation}{exercise}
\counterwithin{equation}{solution}

\definecolor{exercise_color}{RGB}{21,76,121}
\definecolor{exercise_color_fill}{RGB}{252,248,227}

\newcommand{\theexerciseref}{No reference}
\newcounter{reminderequationcounter}
\newcounter{exerciseequationcounter}
\newcounter{solutionequationcounter}


\makeatletter
\newenvironment{reminder}
{
\global\chardef\dc@currentequation=\value{equation}%
\let\c@equation\c@solutionequationcounter
\renewcommand{\theequation}{R\arabic{equation}}%
% comment the following line if you don't use hyperref
\renewcommand{\theHequation}{R\arabic{equation}}%
\refstepcounter{reminder}
\par\vspace{5pt}\noindent
{\Huge\textbf{Reminder}}%
\par\vspace{10pt}\noindent\rmfamily}%
{\par\noindent\hrulefill\vrule width10pt height2pt depth2pt\par\setcounter{equation}{\dc@currentequation}}
\makeatother

\makeatletter
\newenvironment{exercise}[2][\texorpdfstring{\unskip}{}]
{
\global\chardef\dc@currentequation=\value{equation}%
\let\c@equation\c@exerciseequationcounter
\renewcommand{\theequation}{E\arabic{exercise}.\arabic{equation}}%
% comment the following line if you don't use hyperref
\renewcommand{\theHequation}{E\arabic{exercise}.\arabic{equation}}%
\refstepcounter{exercise}
\def\@currentlabel{{#2}}
\label{ref-exercise-\theexercise}
\addcontentsline{toc}{subsubsection}{{#2} #1}
\noindent
\flushleft
\begin{tikzpicture}
    \draw[very thick,exercise_color] (0,0) -- ++(0,+7.5pt)
    -- ++(\textwidth,0) node[midway,above] {\textbf{Exercise #2: #1}}
    -- ++(0,-7.5pt);
\end{tikzpicture}
\vspace{-.3cm}
\begin{tcolorbox}[
    blanker,
    width=\textwidth,
    breakable]
}
{   \end{tcolorbox}
\setcounter{equation}{\dc@currentequation}
\vspace{-.4cm}
\flushleft
\begin{tikzpicture}
    \draw[very thick,exercise_color] (0,0) -- ++(0,-7.5pt)
    -- ++(\textwidth,0)
    -- ++(0,+7.5pt);
\end{tikzpicture}
}
\makeatother

\makeatletter
\ifcsname AvecSolutions\endcsname
\newenvironment{solution}
{
\global\chardef\dc@currentequation=\value{equation}%
\let\c@equation\c@solutionequationcounter
\renewcommand{\theequation}{S\arabic{solution}.\arabic{equation}}%
% comment the following line if you don't use hyperref
\renewcommand{\theHequation}{S\arabic{solution}.\arabic{equation}}%
\refstepcounter{solution}
\noindent
\begin{tcolorbox}[
    colframe=exercise_color,
    colback=exercise_color_fill,
    coltitle=exercise_color_fill,
    title=\centering\textbf{{\hypersetup{allcolors=white} Solution to exercise  \ref{ref-exercise-\theexercise}:}},
    breakable,
    width=\textwidth]
}
{   \end{tcolorbox}
\setcounter{equation}{\dc@currentequation}
}
\else
\newenvironment{solution}{\comment}{\endcomment}
\fi
\makeatother

% --------------------------------------------------------------
% Code environments
% --------------------------------------------------------------
\usemintedstyle{borland}
\providecommand*{\listingautorefname}{Listing}

\newenvironment{python}
{\VerbatimEnvironment
\begin{minted}[
linenos,
% fontfamily=courier,
fontsize=\normalsize,
xleftmargin=21pt,
]{python}}
{\end{minted}}

\newcommand\py[1]{\mintinline{python}{#1}}

\newcommand\la[1]{\mintinline{latex}{#1}}

% Vertical line in matrices

\makeatletter
\renewcommand*\env@matrix[1][*\c@MaxMatrixCols c]{%
  \hskip -\arraycolsep
  \let\@ifnextchar\new@ifnextchar
  \array{#1}}
\makeatother

% Double underline
\def\doubleunderline#1{\underline{\underline{#1}}}


% Multiple autoref: https://tex.stackexchange.com/a/183682

\makeatletter

% define a macro \Autoref to allow multiple references to be passed to \autoref
\newcommand\Autoref[1]{\@first@ref#1,@}
\def\@throw@dot#1.#2@{#1}% discard everything after the dot
\def\@set@refname#1{%    % set \@refname to autoefname+s using \getrefbykeydefault
    \edef\@tmp{\getrefbykeydefault{#1}{anchor}{}}%
    \xdef\@tmp{\expandafter\@throw@dot\@tmp.@}%
    \ltx@IfUndefined{\@tmp autorefnameplural}%
         {\def\@refname{\@nameuse{\@tmp autorefname}s}}%
         {\def\@refname{\@nameuse{\@tmp autorefnameplural}}}%
}
\def\@first@ref#1,#2{%
  \ifx#2@\autoref{#1}\let\@nextref\@gobble% only one ref, revert to normal \autoref
  \else%
    \@set@refname{#1}%  set \@refname to autoref name
    \@refname~\ref{#1}% add autoefname and first reference
    \let\@nextref\@next@ref% push processing to \@next@ref
  \fi%
  \@nextref#2%
}
\def\@next@ref#1,#2{%
   \ifx#2@ and~\ref{#1}\let\@nextref\@gobble% at end: print and+\ref and stop
   \else, \ref{#1}% print  ,+\ref and continue
   \fi%
   \@nextref#2%
}

\makeatother
